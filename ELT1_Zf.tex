\documentclass[a4paper,12pt,]{article}

\usepackage[ngerman]{babel}
\usepackage[T1]{fontenc}
\usepackage{amsmath}
%\usepackage{amsfonts}

\title{ELT1 Zusammenfassung Vorlesungen}
\author{P. Widmer}

\begin{document}

\maketitle
\tableofcontents
\newpage

\section{Übersicht}
In diesem Dokument mache ich eine Zusammenfassung für die ELT1 Vorlesungen. Behandelt werden die einzelnen Themen und Zusammenhänge wie Formeln, Bilder oder sonstige Ausschnitte. Dieses Dokument soll als Nachschlagswerk dienen.

\subsection{Formeln für Grundlagen}
In diesem Abschnitt werden die Zusammenhänge der Einheiten für die Elektrotechnik angeschaut

\subsubsection{el. Stromstärke}
Die elektrische Stromstärke ist definiert durch die aufgewandte Ladung pro Zeiteinheit\\
\underline{Stromstärke $I$}\\ 
Definition:
\begin{equation}
I = \frac{\Delta Q}{\Delta t} = {J}*{A}
\end{equation}
\underline{Widerstand $R$}\\
Geometrische Widerstandsberechnung\\
\begin{equation}
R = \frac{l}{\sigma*A} = \frac{\rho*l}{A}
\end{equation}
$\sigma$ sigma ist die Leitfähigkeit in S/m, daher wird der Widerstand kleiner, je grösser die Leitfähigkeit ist\\
$rho$ rho ist der spezifische Widerstand in $\Omega$m. Der Widerstand wird grösser, je grösser der Widerstand.
\underline{Stromdichte $J$}\\
Definition:
\begin{equation}
J = \frac{I}{A}
\end{equation}\\
\underline{Spannung $U$}\\
Definition:\\
Die elektrische Spannung ist eine Potentialdifferenz und normalisierte potentielle Arbeit. Sie ist definiert duch die aufgewendete Arbeit pro Ladungsmenge:
\begin{equation}
U = \frac{\Delta W}{Q} = \frac{P*t}{I*t} = \frac{P}{I} = R*I = \frac{I}{G}
\end{equation}\\
In Einheiten:
\begin{equation}
V = \frac{Nm}{As} = \frac{kg*m/s^2*m}{As} =\frac{kg*m^2}{A*s^3}
\end{equation}
\underline{Leistung $P$}\\
Definition:\\
BLABLABLA

\subsection{Symbole der Elektrotechnik}

$\sigma$ (Sigma) : Leitfähigkeit in $\sigma$ = $\frac{S}{m}$\\
$\phi$ (Phi) : Potential in $V$\\
$\rho$ (Rho) : Spezifischer Widerstand in $\Omega$m

\subsection{Wichtige Zahlen (auswendig verlangt)}

Ladung eines Elektrons:
kleinste Ladungsmenge: $1.6*10^{-19} C$\\
Leitfähigkeit und Spezifischer Widerstand von Kupfer:

\end{document}